\documentclass{article}
\usepackage{amsmath}
\usepackage[UTF8]{ctex}
\usepackage{anyfontsize}
\usepackage{fontspec}
\title{标题}
\author{作者}
\date{2023/8/20}
\begin{document}
\maketitle
\tableofcontents
\newpage

\section{一级标题}
一级标题一级标题一级标题一级标题一级标题一级标题一级标题一级标题一级标题一级标题一级标题
一级标题一级标题一级标题一级标题一级标题一级标题一级标题一级标题一级标题一级标题一级标题
一级标题一级标题
\subsection{二级标题}
二级标题二级标题二级标题二级标题二级标题二级标题二级标题二级标题二级标题二级标题二级标题
二级标题二级标题二级标题二级标题二级标题二级标题二级标题二级标题二级标题二级标题二级标题
二级标题二级标题
\subsubsection{三级标题}
三级标题三级标题三级标题三级标题三级标题三级标题三级标题三级标题三级标题三级标题三级标题
三级标题三级标题三级标题三级标题三级标题三级标题三级标题三级标题三级标题三级标题三级标题
三级标题三级标题

\section{字体}
\subsection{字体大小}
\noindent
\Huge 
大小 
\huge 
大小 
\LARGE 
大小 
\Large 
大小 
\large 
大小 
\normalsize 
大小 
\small 
大小  
\footnotesize 
大小  
\scriptsize 
大小  
\tiny 
大小 
\normalsize

\section{段落}
\paragraph{}
这是段落1这是段落1这是段落1这是段落1这是段落1这是段落1这是段落1这是段落1这是段落1
这是段落1这是段落1这是段落1这是段落1这是段落1这是段落1
\paragraph{}
这是段落2这是段落2这是段落2这是段落2这是段落2这是段落2这是段落2这是段落2这是段落2
这是段落2这是段落2这是段落2这是段落2这是段落2这是段落2
\subsection{副段落}
\subparagraph{}
这是副段落1这是副段落1这是副段落1这是副段落1这是副段落1这是副段落1这是副段落1这是副段落1
这是副段落1这是副段落1这是副段落1这是副段落1

\section{添加数学公式}
\subsection{插入行内公式}
Einstein 's $E=mc^2$.
\subsection{插入行间公式}
\paragraph{}
文本文本文本文本文本文本文本文本文本文本文本文本文本文本文本文本文本文本文本文本
文本文本文本文本
\[ E=mc^2. \]
\[ E=mc^2. \]
\[ E=mc^2. \]
文本文本文本文本文本文本文本文本文本文本文本文本文本文本文本文本文本文本文本文本
文本文本文本文本
\subsection{在一行中插入多个公式}
\begin{displaymath}
    S_{n+1}=S_{n}+S_{n},
    S_{n}=2^{n}
\end{displaymath}
\subsection{公式编辑}
\subsubsection{上下标}
\Large
$a^{n}$
$a_{n}$
\subsubsection{分式}
$\frac{m}{n}$
\subsubsection{开方}
$\sqrt{x}$
$\sqrt[n]{x}$
\subsubsection{累计求和}
$\sum_{i=m}^{n}$
\subsubsection{积分}
$\int_{i=m}^{n}$
\subsubsection{向量}
$\vec a$
$\overrightarrow{AB}$
\subsubsection{省略号}
$a+b+\cdots+z$
\subsubsection{大括号}
$\underbrace{a+b+\cdots+z}_{26}$
\subsubsection{横杠}
$\overline{m+n}$
$\underline{m+n}$

\end{document}